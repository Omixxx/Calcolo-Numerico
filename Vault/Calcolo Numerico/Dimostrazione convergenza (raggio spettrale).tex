% Options for packages loaded elsewhere
\PassOptionsToPackage{unicode}{hyperref}
\PassOptionsToPackage{hyphens}{url}
%
\documentclass[
]{article}
\usepackage{amsmath,amssymb}
\usepackage{iftex}
\ifPDFTeX
  \usepackage[T1]{fontenc}
  \usepackage[utf8]{inputenc}
  \usepackage{textcomp} % provide euro and other symbols
\else % if luatex or xetex
  \usepackage{unicode-math} % this also loads fontspec
  \defaultfontfeatures{Scale=MatchLowercase}
  \defaultfontfeatures[\rmfamily]{Ligatures=TeX,Scale=1}
\fi
\usepackage{lmodern}
\ifPDFTeX\else
  % xetex/luatex font selection
\fi
% Use upquote if available, for straight quotes in verbatim environments
\IfFileExists{upquote.sty}{\usepackage{upquote}}{}
\IfFileExists{microtype.sty}{% use microtype if available
  \usepackage[]{microtype}
  \UseMicrotypeSet[protrusion]{basicmath} % disable protrusion for tt fonts
}{}
\makeatletter
\@ifundefined{KOMAClassName}{% if non-KOMA class
  \IfFileExists{parskip.sty}{%
    \usepackage{parskip}
  }{% else
    \setlength{\parindent}{0pt}
    \setlength{\parskip}{6pt plus 2pt minus 1pt}}
}{% if KOMA class
  \KOMAoptions{parskip=half}}
\makeatother
\usepackage{xcolor}
\setlength{\emergencystretch}{3em} % prevent overfull lines
\providecommand{\tightlist}{%
  \setlength{\itemsep}{0pt}\setlength{\parskip}{0pt}}
\setcounter{secnumdepth}{-\maxdimen} % remove section numbering
\ifLuaTeX
  \usepackage{selnolig}  % disable illegal ligatures
\fi
\IfFileExists{bookmark.sty}{\usepackage{bookmark}}{\usepackage{hyperref}}
\IfFileExists{xurl.sty}{\usepackage{xurl}}{} % add URL line breaks if available
\urlstyle{same}
\hypersetup{
  hidelinks,
  pdfcreator={LaTeX via pandoc}}

\author{}
\date{}

\begin{document}

Sfruttiamo la definizione di errore per per mostrare che esso tende a
\(0\) visto che \(T^{(k)}\) tende a \(0\) per \(k\to \infty\) se
\(\rho(T)<1\)

partiamo definendoci l'errore assoluto \(e^{(k+1)}\) \[
\begin{align}e^{(k+1)}=x-x^{(k+1)}=(Tx+c)-(Tx^{(k)}+c)=\newline=Tx{\cancel{+c}}-Tx^{(k)}{\cancel{{-c}}}=Tx+Tx^{(k)}=  &  \\= T(x-x^{(k)})=Te^{(k)},\quad k=1,2,\ldots\end{align}
\]

e applichiamo questa definizione ricorsivamente partendo da \(0\) \[
\begin{aligned}
&e^{(0)}&& =x-x^{(0)}  \\
&e^{(1)}&& =Te^{(0)}  \\
&e^{(2)}&& =Te^{(1)}=TTe^{(0)}=T^{2}e^{(0)}  \\
&e^{(3)}&& =Te^{(2)}=TT^2e^{(0)}=T^3e^{(0)}  \\
&&&...&   \\
&e^{(k+1)}&& =Te^{(k)}=TT^{k}e^{(0)}=T^{k+1}e^{(0)} 
\end{aligned}
\] ottenendo questa importante relazione
\(e^{(k+1)} = Te^{(k)} = T^{k+1}e^{(0)}\)

Ora possiamo usare il seguente {lemma} \[ 
 \boxed{ \lim_{m\to\infty}B^m=0\Leftrightarrow\rho(B)<1 }
\] per enunciare il teorema che segue

\begin{theorem}
$\forall x^{(0)} \text{ in } \mathbb{R}^{n} \text{ la successione } x^{(k)}, k=0\dots \text{ definita da}$ 
$$x^{(k+1)}=Tx^{(k)}+c\quad\forall k\geq0\text{ e }c\neq0$$
dato il lemma e il modo in cui abbiamo definito l'errore possiamo dire che $T^{k+1}e^{(0)}=0 \text{ per } k \to \infty$
e visto che l'errore tenderà ad annullarsi con l'aumentare delle iterate possiamo dire che il metodo converge all'unica soluzione  $x=Tx+c \iff \rho(T) <1$ 
\end{theorem}

\end{document}
